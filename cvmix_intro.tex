\chapter{\scshape CVMix Parameterizations}
\label{chapter:cvmix_intro}

\minitoc
\vspace{.5cm}

We provide in this chapter an overview of the vertical mixing
parameterizations available with CVMix.


\section{Vertical mixing parameterizations in CVMix}
\label{section:vert_mix_schemes_cvmix}

The CVMix Project aims to address the needs of various ocean modeling
groups to code, test, tune, and document parameterizations of oceanic
vertical mixing for numerical ocean simulations.  The project focuses
on first-order turbulence closures for vertical mixing
processes.\footnote{Those interested in higher order turbulence
  closure schemes for ocean modeling may with to consider using the
  General Ocean Turbulence Model (GOTM) from \cite{GOTM}.}  CVMix does
{\it not} determine time stepping for the model prognostic fields.
Instead, time stepping is the responsibility of the calling model
code.  The following schemes are included in the CVMix
parameterizations during phase I of the development.

\begin{itemize}

\item {\sc Static background mixing} (Chapter
  \ref{chapter:cvmix_background}): Certain turbulent processes, in
  particular the ambient background gravity wave ``noise'', constitute
  a background level of mixing that is largely steady in time from the
  pespective of large-scaling ocean modeling.  Though assumed to be
  time independent, these processes generally have a nontrivial space
  dependence.  CVMix provides options for various of these time
  independent schemes:
\begin{itemize}
\item the vertical profile from \cite{BryanLewis1979};
% \item the profile from \cite{Henyey_etal1986} in which mixing near the
%   equator is very small, as measured by \cite{Gregg_etal2003}; 
% \item the approaches from \cite{Jochum2009}, in which enhanced mixing
%   from parametric subharmonic mixing is included. 
\end{itemize}


\item {\sc Shear induced mixing} (Chapter \ref{chapter:cvmix_shear}):
  The following schemes are available for shear mixing:
 \begin{itemize}
 \item \cite{PPvmix}, applicable largely for tropical circulation;
 \item \cite{LargeKPP} and \cite{Large_Gent1999}, which builds on the
   \cite{PPvmix} scheme;
  \end{itemize}


\item {\sc Tidally induced mixing} (Chapter
  \ref{chapter:cvmix_tidal}): The following schemes are available for
  parameterizing mixing induced by ocean tides.
  \begin{itemize}
   \item \cite{Simmonsetal2004} 
   \item \cite{Melet_etal_2013}
\end{itemize}


\item {\sc Double diffusive processes} (Chapter
  \ref{chapter:cvmix_ddiffusion}): Double diffusive processes arise
  from the distinct mixing properties of temperature relative to
  salinity and other material tracers.

\item {\sc KPP surface boundary layer} (Chapter
  \ref{chapter:cvmix_kpp}): The K-profile parameterization (KPP)
  scheme from \cite{LargeKPP} provides for a diffusivity as well as a
  non-local transport, each within the surface planetary boundary
  layer.

\item {\sc Vertical convective mixing}: Vertical profiles can become
  gravitationally unstable, such as when the ocean is forced with a
  negative buoyancy flux.  Older approaches such as \cite{CoxModel}
  and \cite{Rahmstorf1993} considered a convective {\it adjustment}
  algorithm, in which vertical pairs of grid cells were adjusted
  towards a profile of static stability.  In effect, the vertical
  diffusivity is infinite when using adjustment schemes.  CVMix does
  {\it not} provide options for convective adjustment.  Instead, CVMix
  allows for the specification of a diffusivity that is large in
  regions of gravitational instability, thus enabling vertical
  convective {\it mixing} rather than {\it adjustment}.  Notably, when
  using the KPP surface boundary layer scheme, convective mixing is
  {\it not} computed inside the KPP boundary layer.  Instead, it is
  only computed beneath the boundary layer, and it is done so {\it
    after} the KPP boundary layer matching has occurred (see Section
  \ref{section:vert_mix_schemes_ordering_cvmix}).


\end{itemize}


\section{General form of CVMix parameterizations}
\label{section:vert_mix_schemes_general_cvmix}

CVMix focuses only on vertical transport processes in the ocean, so
that we are concerned with the tracer or velocity equation in the form
\begin{equation}
  \frac{\partial \overline{\lambda} }{\partial t}  = 
 - \frac{\partial}{\partial z}  \left( \overline{w' \, \lambda'} + \overline{w} \, \overline{\lambda} \right). 
\end{equation}
In this equation, $w'$ is the turbulent or fluctuating portion of the
vertical velocity\footnote{In Chapter \ref{chapter:cvmix_kpp}, we
  follow the notation of \cite{LargeKPP} by writing the mean
  quantities with an uppercase, $W$ and $\Lambda$, and turbulent
  fluctuations with a lowercase, $w$ and $\lambda$. For the present
  chapter, we follow the more standard notation of equation
  (\ref{eq:mean-turbulent-decompose}).}
\begin{equation}
 w = w' + \overline{w},
\label{eq:mean-turbulent-decompose-w}
\end{equation}
$\lambda'$ is a fluctuating scalar or velocity component
\begin{equation}
 \lambda = \lambda' + \overline{\lambda},
\label{eq:mean-turbulent-decompose-lambda}
\end{equation}
and the overline denotes an Eulerian ensemble or time average that
separates the mean flow from turbulent fluctuations.  The vertical
flux $\overline{w} \, \overline{\lambda}$ is represented in a
numerical model by an advection operator or remapping operation, and
it is comprised of flow {\it resolved} by the model grid.  In
contrast, $\overline{w' \, \lambda'}$ is the correlation between
fluctuations in the vertical velocity component and the field
$\lambda$. This correlation is often termed the {\it turbulent flux}
of $\lambda$.  It is not explicitly represented by a numerical model,
since it arises from processes at the subgrid scale.  CVMix code
provides a suite of closures, or {\it parameterizations}, for this
subgrid scale flux.

We are interested in those correlations $\overline{w' \, \lambda'}$
that can be parameterized in terms of vertical diffusion or vertical
non-local mixing.  That is, all parameterizations considered in CVMix
can be formulated in terms of a diffusivity and a non-local transport,
in which case the turbulent flux is written as
\begin{equation}
  \overline{w' \, \lambda'} = 
  -K_{\lambda} \left( \frac{\partial \overline{\lambda}}{\partial z} - \gamma_{\lambda} \right).
\label{eq:cvmix-parameterization}
\end{equation}
The first term on the right hand side of equation
(\ref{eq:cvmix-parameterization}) provides for the familiar
downgradient vertical diffusion determined by a non-negative vertical
diffusivity, $K_{\lambda} \ge 0$, and the local vertical derivative of
the model's resolved field, $\partial \overline{\lambda} / \partial
z$.  This term is referred to as the local portion of the vertical
mixing parameterization
\begin{equation}
\overline{ w' \, \lambda' }^{\, \mbox{\scriptsize local}} = -K_{\lambda} \left( \frac{\partial \overline{\lambda}}{\partial z} \right).
\end{equation}
Note that the term ``local'' is used for this portion of the
parameterized flux (\ref{eq:cvmix-parameterization}) since it is
determined by the local derivative of the mean field,
$\overline{\lambda}$.  However, the diffusivity can generally be
determined as a non-local function of boundary layer properties, with
such being the case for the KPP scheme (Chapter
\ref{chapter:cvmix_kpp}).  The second term in equation
(\ref{eq:cvmix-parameterization}), $\gamma_{\lambda}$, accounts for
non-local transport that is not directly associated with local
vertical gradients of $\lambda$, in which we have
\begin{equation}
\overline{w' \, \lambda'}^{\, \mbox{\scriptsize non-local}} = K_{\lambda} \; \gamma_{\lambda}.
\end{equation}
KPP is the only scheme available with CVMix that prescribes a nonzero
value for $\gamma_{\lambda}$.  

Every scheme available in CVMix computes parameterized turbulent
fluxes based on a suite of inputs from the calling model, such as the
surface buoyancy and momentum fluxes, vertical stratification, and
vertical shear.  There may also need to be information about the
bottom roughness and unresolved tide speeds.  Besides the diffusivity
and non-local transport, various diagnostic fields are available to
help understand the internal workings of the parameterizations.



\section{Ordering the calculations of CVMix parameterizations}
\label{section:vert_mix_schemes_ordering_cvmix}

Certain of the CVMix schemes are independent, with their resulting
diffusivities and viscosities merely added to the total mixing
coefficients.  Other schemes, however, must be called in a certain
order given the underlying assumptions built into the scheme.  The
main issue concerns the KPP scheme, and there are two points to
consider. 
\begin{itemize}
\item {\sc KPP after interior non-convective mixing}: Since the KPP
  scheme matches diffusivities at the base of the boundary layer to
  values computed beneath the boundary layer (Section
  \ref{subsection:kpp-shape-function}), KPP must be called subsequent
  to those schemes determining non-convective mixing coefficients in
  the ocean interior.

\item {\sc KPP before interior convective mixing}: The matching of
  diffusivities at the base of the KPP boundary layer intrinsically
  assumes there to be a transition from typically larger diffusivities
  in the boundary layer to typically smaller diffusivities in the
  interior.  However, this sort of transition cannot always be
  ensured, since gravitationally unstable water can appear beneath the
  boundary layer in which case the interior diffusivities can be quite
  large.  Problems with the diffusivity matching occur if insisting
  that KPP match its boundary layer diffusivity to a potentially large
  interior diffusivity arising from convective mixing.  To eliminate
  these problems, parameterized convective mixing in the ocean
  interior must be called {\it after} the KPP boundary layer scheme.
  Furthermore, note that convective parameterizations are not used
  inside the KPP boundary layer, since KPP provides the mixing
  coefficients in the boundary layer.

\end{itemize}
These considerations lead to the recommended flow diagram shown in
Figure \ref{fig:vertical_mix_flow_cvmix} for use of CVMix schemes.

%%%%%%%%%%%%%%%%%%%% %%%%%%%%%%%%%%%%%%%%%%%%%
\begin{figure}[h!t]
\rule{\textwidth}{0.005in}
\begin{center}
\includegraphics[angle=0,width=15cm]{./mfpic_figs/cvmix_flow_diagram.pdf}
\caption[Flow diagram for CVMix schemes]{\sf This flow diagram depicts
  the general algorithmic steps required to utilize the CVMix
  parameterization modules.  The initialization step occurs in one of
  the CVMix drivers that exercises a chosen mixing scheme {\tt
    driver.F90} if running CVMix code as a stand-alone one-dimensional
  model.  Otherwise it occurs via the ocean model if running CVMix
  modules as part of an ocean model such as MPAS-ocean, MOM, or POP.
  This initialization serves to set up arrays and derived type
  structures, all as a function of the input that it receives from the
  calling ocean model code.  The next step during initialization is to
  call the module {\tt cvmix\_background.F90} to fill chosen static
  background diffusivities.  Upon entering the time dependent portion
  of the ocean model integration, the driver receives surface fluxes
  and penetrative radiative fluxes.  Calls are made to chosen interior
  non-convective mixing schemes, such as shear mixing, tide mixing,
  and double diffusion.  Thereafter, the surface boundary layer scheme
  is called, with KPP the scheme targetted for CVMix.  The boundary
  layer calculation is key to the whole process, as it must come after
  the interior non-convective portion, and before the convective
  portion.  After the boundary layer, then convective mixing is
  called, with regions of gravitationally unstable water given a large
  diffusivity.  Notably, if KPP is used for the surface boundary
  layer, parameterized convective mixing is performed only beneath the
  KPP boundary layer.  The final step returns the diffusivity
  $K_{\lambda}$, viscosity, and non-local transport
  $\gamma_{\lambda}$, arrays to the calling ocean model code.  A new
  time step starts by reinitializing the diffusivities to their static
  background values.}
\label{fig:vertical_mix_flow_cvmix}
\end{center}
\rule{\textwidth}{0.005in}
\end{figure}
%%%%%%%%%%%%%%%%%%%%%%%%%%%%%%%%%%%%%%%%%%%%%%%%%%%%%%%%%%%%%%%%%%%%%%%%



