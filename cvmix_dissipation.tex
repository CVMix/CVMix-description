\chapter{\scshape Diffusivity based on a chosen dissipation}
\label{chapter:cvmix_dissipate}

\minitoc
\vspace{.5cm}

The purpose of this chapter is to summarize a method that is not
available in CVMix, yet which may be of interest to modellers using
CVMix schemes.  This method specifies vertical tracer diffusivities
based on setting a floor to the power dissipation.  This approach was
found to be useful in the ESM2G earth system model documented by 
\cite{Dunne_etal_part1_2012}.


\section{Power dissipation from vertical diffusion}
\label{section:vert_dissipation_formulation}

Vertical tracer diffusion is associated with a dissipation of power.
Assuming temperature and salinity have the same vertical diffusivities
leads to the expression for power dissipation
($\mbox{W}~\mbox{m}^{-3}$)
\begin{equation}
\begin{split}
 \epsilon &= \rho \,  \kappa \, N^{2}
 \\
&= -\kappa \, g \, \left( \frac{\partial \rho}{\partial \theta} \, \frac{\partial \theta}{\partial z} 
                                      +\frac{\partial \rho}{\partial S}         \, \frac{\partial S}{\partial z}
                                \right).
\end{split}
\end{equation}
In these equations, $\kappa$ is the vertical tracer diffusivity and
$g$ is the gravitational acceleration. When the temperature and
salinity diffusivities differ, as occurs with double diffusion
(Chapter \ref{chapter:cvmix_ddiffusion}), power dissipation is computed
via
\begin{equation}
\begin{split}
  \epsilon &= 
 -g \, \kappa{\mbox{\tiny temp}} \,  \left( \frac{\partial \rho}{\partial \theta} \, \frac{\partial \theta}{\partial z} \right)
-g\, \kappa{\mbox{\tiny salt}}     \, \left(  \frac{\partial \rho}{\partial S}        \, \frac{\partial S}{\partial z} \right).
\end{split}
\end{equation}


\section{Setting a floor to the dissipation} 
\label{section:setting-a-floor-to-dissipation}


We now compute a floor to the dissipation according to 
\begin{equation}
 \epsilon_{\mbox{\tiny floor}} =  \epsilon_{\mbox{\tiny min}} + B \, |N|, 
\end{equation}
 where 
\begin{equation}
  \epsilon_{\mbox{\tiny min}} \sim 10^{-6}~\mbox{W}~\mbox{m}^{-3}
\end{equation}
is a specified minimum power dissipation (set according to a
namelist), $B$ is another namelist parameter (physical dimensions
$\mbox{J}~\mbox{m}^{-3}$) further discussed below, and $|N|$ is the
absolute value of the buoyancy frequency.  As discussed below (see
equation (\ref{eq:gargett-scaling})), the $B \, |N|$ contribution to
dissipation is motivated by the stratification dependent diffusivity
proposed by \cite{Gargett1984}.

We establish a floor to the vertical diffusivity according to
\begin{equation}
\begin{split}
  \kappa_{\mbox{\tiny floor}} &= \frac{ \epsilon_{\mbox{\tiny floor}}  \, \Gamma^{\mbox{\tiny regularized}}}{\rho \, N^{2}} 
  \\
 &\approx \frac{\epsilon_{\mbox{\tiny floor}} \, \Gamma_{o}  }{\rho_{o} \, (N^{2} + \Omega^{2})}.
\end{split}
\end{equation}
 In this equation, 
\begin{equation}
 \Gamma^{\mbox{\tiny regularized}} = \Gamma_{o} \,  \left( \frac{ N^{2}}{ N^{2} +\Omega^{2}} \right)
\end{equation}
is a regularized mixing efficiency introduced by
\cite{Melet_etal_2013},
\begin{equation}
 \Gamma_{o} = 0.2
\end{equation}
is a nominal value for stratified water where $N^{2} >> \Omega^{2}$,
and
\begin{equation}
 \Omega = 7.2921 \times 10^{-5} \mbox{s}^{-1}
\end{equation} 
is the angular rotation rate of the earth about its axis and around
the sun (see also equation (\ref{eq:Omega-defined})).  In the special
case of $N^{2} >> \Omega^{2}$, and $\epsilon_{\mbox{\tiny floor}}
\approx B \, |N|$, then
\begin{equation}
 \kappa_{\mbox{\tiny floor}} \approx \left( \frac{ \Gamma_{o} \,  B }{\rho_{o}} \, \right) \, |N|^{-1}.
\label{eq:gargett-scaling}
\end{equation} 
This scaling with respect to buoyancy frequency was suggested by
\cite{Gargett1984}, where she recommended in open water to choose
\begin{equation}
   \frac{ \Gamma_{o} \,  B }{\rho_{o}} \approx 10^{-7}~\mbox{m}^{2}~\mbox{s}^{-2},
\end{equation}
 so that 
\begin{equation}
 B \approx 5 \times  10^{-4}~\mbox{J}~\mbox{m}^{-3}.
\end{equation}
This value may in fact be quite large, with the value $B \sim 1.5
\times 10^{-4}~\mbox{J}~\mbox{m}^{-3}$ used in the isopycnal model of
\cite{Dunne_etal_part1_2012}.

When utilizing this method, the tracer diffusivity used for
temperature, salinity, and passive tracers is set to be no smaller
than $\kappa_{\mbox{\tiny floor}}$.  The check should be made at the
end of the vertical mixing processes for whether the diffusivity
satisfies this constraint (see Figure
\ref{fig:vertical_mix_flow_cvmix}).  If too small, then diffusivity is
increased to meet the constraint.
