\chapter{\scshape Sample chapter for CVMix documentation}
\label{chapter:cvmix_MODULE}

\minitoc
\vspace{.5cm}

\begin{mdframed}[backgroundcolor=lightgray!50]
  This chapter provides a sample for what is useful to include in the
  CVMix documentation of a physical parameterization scheme.  Clear
  and complete CVMix documentation is critical for the intelligent use
  of any scheme, so the author should focus on creating documentation
  that is useful and pedagogical.  The following CVMix Fortran module
  is directly connected to the material in this chapter:
\begin{align*} 
 &{\tt cvmix\_MODULE.F90}
\end{align*}
\end{mdframed}

Here are some suggestions for writing this document.
\begin{itemize}

\item Aim for a level of concise pedagogy, so that the reader will
  readily understand the scheme, but without going into too many
  details that are available in published literature.  Some of the
  other CVMix chapters are good examples of this philosophy (e.g.,
  tide mixing chapter \ref{chapter:cvmix_tidal}), though others are
  not (e.g., KPP chapter \ref{chapter:cvmix_kpp}).

\item This chapter should describe any test cases available from the
  scheme, including sample figures.  Test cases allow the new user to
  verify a particular implementation, and to provide a sample of how
  the physical parameterizion works.

\item Add new bibliography entries to the .bib file in the directory
  {\tt cvmix\_manual/references}.

\item Figures associated with this document should ideally be named
  {\tt new\_scheme\_figure\_number.pdf} or the like, thus making it
  simpler to associate a figure with a chapter.

\end{itemize}



\section{Introduction to the mixing scheme}
\label{section:intro_new_scheme}

This section provides a broad overview of the scheme, focusing on the
scientific background and the processes available from the CVMix
module.  


\section{Theory}
\label{section:theory_new_scheme}

This section develops the basic theory for the scheme, summarizing
elements of the published literature in a manner that provides an
intellectual foundation for the CVMix scheme.  The reader should
understand what the scheme aims to do from a physics perspective.  

\section{Numerical implementation in CVMix}
\label{section:numerics_new_scheme}

This section develops the numerical implementation choices made for
the CVMix implementation.  Please include here a full description of
options for using the scheme, as well as a list of parameters and
their physical dimensions.  Recommended usage and test cases can be
presented here as well.  


\section{Further sections}

The above template is clearly insufficient for some purposes.  Feel
free to modify as you see appropriate.  


