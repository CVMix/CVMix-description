\chapter{\scshape Parameterized shear induced mixing}
\label{chapter:cvmix_shear}

\minitoc
\vspace{.5cm}

In this chapter we detail the CVMix implementations of
parameterizations arising from shear induced mixing in stratified
flows.  The following CVMix Fortran module is directly connected to
the material in this chapter:
\begin{align*} 
 &{\tt cvmix\_shear.F90}
\end{align*}


\section{Mixing from shear instability}
\label{section:shear-instability-mixing}

Shear induced mixing occurs when vertical shears in the horizontal
velocity overcome the stabilizing effects from vertical buoyancy
stratification.  Shear instability is governed by the local or
gradient Richardson number discussed in Section
\ref{section:gradient-richardson-number-elements}
\begin{equation}
 \mbox{Ri} = \frac{N^{2}}{|\partial_{z} \, {\bf u}|^{2} }.
\label{eq:richardson-number}
\end{equation}
In this expression, $N^{2}$ is the squared buoyancy frequency (Section
\ref{section:buoyancy-frequency-elements}), with a linear
approximation given by
\begin{equation}
 N^{2}_{\mbox{\scriptsize linear}} 
 = g \, \left( \alpha \, \frac{\partial \Theta}{\partial z} 
                         - \beta  \, \frac{\partial S}{\partial z} \right).
\label{eq:buoyancy-frequency-shear}
\end{equation}
We assume throughout that the water is stably stratified, so that
$N^{2} > 0$.  With $N^{2} < 0$, one should set the diffusivity and
viscosity to a large value to parameterize mixing associated with the
gravitational instability (see Chapter
\ref{chapter:cvmix_convection}).

The denominator in the Richardson number in equation
(\ref{eq:richardson-number}) is the squared vertical shear of the
horizontal velocity vector resolved by the model grid
\begin{equation}
 |\partial_{z} \, {\bf u}|^{2}  =
  \left( \frac{\partial u}{\partial z} \right)^{2} 
 +
 \left( \frac{\partial V}{\partial z} \right)^{2}. 
\end{equation}
When the Richardson number is below a critical value, $\mbox{Ri} <
\mbox{Ri}_{c}$, shear instabilities can grow to initiate turbulence,
which in turn leads to enhanced mixing.

The canonical value of the critical gradient Richardson number is
\begin{equation}
 \mbox{Ri}_{c} =1/4   \qquad \mbox{analytical value from shear layer instability.} 
\end{equation}
This value corresponds to the critical value for initiation of a
Kelvin-Helmholz instability in a shear layer \citep{miles1961}.
However, as reviewed in Section 4b of \cite{Jacksonetal2008}, there
are many reasons that 1/4 is not always the optimal value to use in
numerical simulations.  Thus, most schemes choose noticeably larger
values, which allows for shear-induced mixing to be intiated in less
unstable regions of the model's resolved flow.




\section{Richardson number mixing from \cite{PPvmix}}
\label{section:shear-instability-parameterized-ppvmix}

An early form for parameterized shear mixing is that proposed by
\cite{PPvmix}, where their application focused on equatorial dynamics.
They used a different viscosity, $\nu_{\mbox{\tiny pp shear}}$, and
diffusivity, $\kappa_{\mbox{\tiny pp shear}}$.  For gravitationally
stable profiles (i.e., $N^{2} > 0$), they chose
\begin{align}
 \nu_{\mbox{\tiny pp shear}}&= \frac{\nu_{0}}{(1 + a  \, \mbox{Ri})^{n}} 
\\
 \kappa_{\mbox{\tiny pp shear}}&= \frac{\nu_{0}}{(1 + a  \, \mbox{Ri})^{n+1}}, 
\end{align}
where $\nu_{0}$, $a$ and $n$ are adjustable parameters.  Common
settings used in POP are $a=5$ and $n=2$.  Note that the Prandtl
number is not constant using this prescription
\begin{subequations}
\begin{align}
  \mbox{Pr}_{\mbox{\scriptsize pp}} &= \frac{\nu_{\mbox{\tiny pp shear}}}{\kappa_{\mbox{\tiny pp  shear}}} 
 \\
 &= 1 + a  \, \mbox{Ri}.
\end{align}
\end{subequations}


\section{Richardson number mixing from \cite{LargeKPP}}
\label{section:shear-instability-parameterized-kpp}

For regions beneath the KPP boundary layer (see Figure
\ref{fig:boundary-layer-schematic-kpp}), \cite{LargeKPP} and
\cite{Large_Gent1999} parameterized shear induced mixing using the
following diffusivities
\begin{equation}
  \kappa_{\mbox{\tiny kpp shear}}= \left\{
\begin{array}{llll}
&\kappa_{0}  &\mbox{Ri} < 0  &\mbox{gravitational instability regime} 
 \\
   &\kappa_{0} \left[1- \left(\frac{\mbox{Ri}}{\mbox{Ri}_0}\right)^2 \right]^3
   &0 < \mbox{Ri}  <  \mbox{Ri}_0   &\mbox{shear instability regime}
\\
 &0 &\mbox{Ri} \ge \mbox{Ri}_{0} &\mbox{stable regime.} 
\end{array}
 \right.
\label{eq:kappa-shear-kpp}
\end{equation}
The form in the shear instability regime falls most rapidly near
$\mbox{Ri} = 0.4 \, \mbox{Ri}_{0}$, which aims to parameterize the
onset of shear instability. In this neighborhood, rapid changes in
$\mbox{Ri}$ can cause gravitational instabilities to develop in the
vertical, but these are largely controlled by vertically smoothing
$\mbox{Ri}$ profiles with a $1-2-1$ smoother.  Unlike \cite{PPvmix},
\cite{LargeKPP} chose a unit Prandtl number for shear induced mixing;
i.e., the shear induced viscosity is the same as the shear induced
diffusivity
\begin{subequations}
\begin{align}
  \mbox{Pr}_{\mbox{\scriptsize kpp}} &= \frac{\nu_{\mbox{\tiny kpp shear}}}{\kappa_{\mbox{\tiny kpp  shear}}} 
 \\
 &= 1.
\end{align}
\end{subequations}


\section{Richardson number mixing from \cite{Jacksonetal2008}}
\label{section:shear-instability-parameterized-jackson-etal}

To be completed. 