\chapter{\scshape Double diffusion}
\label{chapter:cvmix_ddiffusion}

\minitoc
\vspace{.5cm}

\begin{mdframed}[backgroundcolor=lightgray!50]
This chapter details the parameterization of mixing from double
diffusive processes.  The following CVMix Fortran module is directly
connected to the material in this chapter:
\begin{align*} 
 &{\tt cvmix\_ddiff.F90}
\end{align*}
\end{mdframed}


\section{Introduction to mixing from double diffusive processes}

Double diffusion processes \citep{Schmitt1994} have the potential to
significantly enhance vertical diffusivities.  The key stratification
parameter of use for double diffusive processes is
\begin{equation}
 R_{\rho} = \frac{\alpha}{\beta} \left( \frac{\partial\Theta/\partial
     z}{\partial S / \partial z} \right),  
\label{eq:stratification-parameter-ddiffuse}
\end{equation}
 where the thermal expansion coefficient is given by 
\begin{equation}
 \alpha = -\frac{1}{\rho} \, \left( \frac{\partial \rho}{\partial  \Theta} \right),
\end{equation}
 and the haline contraction coefficient is 
\begin{equation}
 \beta = \frac{1}{\rho} \, \left( \frac{\partial \rho}{\partial  S} \right).
\end{equation}
Note that the effects from double diffusive processes on viscosity are
ignored in CVMix for two reasons:
\begin{itemize}
 \item The effects on viscosity are not well known.
 \item For most applications, the vertical Prandtl number is larger
   than unity (often 10) for background viscosities (Chapter
   \ref{chapter:cvmix_background}), so that modifying the vertical
   viscosity according to double diffusion will not represent a
   sizable relative impact.
\end{itemize}


There are two regimes of double diffusive processes, with the
parameterization different in the regimes.  We now detail how CVMix
parameterizes vertical mixing in these two regimes.  


\section{Salt fingering regime}
\label{section:salt-fingering-param}

The salt fingering regime occurs when salinity is destabilizing the
water column (salty above fresh water) and when the stratification
parameter $ R_{\rho}$ is within a particular range:
\begin{align}
  &\frac{\partial S}{\partial z} > 0 
 \\
  &1 < R_{\rho} < R_{\rho}^0    = 2.55.
\end{align}
The parameterized vertical diffusivity in this regime is fit to
observational estimates given by \cite{StLaurentSchmitt1999}, who
propose the following form 
\begin{equation}
\kappa_d = \kappa_d^0 \left[1-\frac{{R_{\rho}-1}}{{R_{\rho}^0 -1}}\right]^3.
\label{eq:kappadsf}
\end{equation}
The default values for the parameter $\kappa_d^0$ are set to
\begin{equation}
   \kappa_d^0   = \left\{  \begin{array}{lll} 
    &1 \times 10^{-4}~\mbox{m}^{2}~\mbox{s}^{-1} &\mbox{for salinity and other tracers} 
  \\
    &0.7 \times 10^{-4}~\mbox{m}^{2}~\mbox{s}^{-1} &\mbox{for temperature.} 
  \end{array}
 \right. 
\end{equation}


\section{Diffusive convective regime}
\label{section:diffusive-convective-param}

Diffusive convective instability occurs where the temperature is
destabilizing (cold above warm) and with $0 < R_{\rho} < 1$
\begin{align}
  &\frac{\partial \Theta}{\partial z} < 0 
 \\
  &0 < R_{\rho} < 1.
\end{align}
For temperature, the vertical diffusivity used in \cite{LargeKPP} is
given by
\begin{eqnarray}
\kappa_d = \nu_{\mbox{\tiny molecular}} \times 0.909
     \exp\left(4.6\exp\left[-.54\left(R_{\rho}^{-1} - 1\right)\right]\right),
\label{eq:kappaddd}
\end{eqnarray}
where 
\begin{equation}
  \nu_{\mbox{\tiny molecular}} = 1.5 \, \times \, 10^{-6}~\mbox{m}^{2}~\mbox{s}^{-1}
\end{equation}
is the molecular viscosity of water.  Multiplying the diffusivity
(\ref{eq:kappaddd}) by the factor
\begin{equation}
\mathrm{factor} = 
\begin{cases}
\left(1.85 - 0.85R_{\rho}^{-1}\right) R_{\rho} & 0.5\leq R_{\rho}<1 \\
0.15  R_{\rho}                                 & R_{\rho} < 0.5, 
\end{cases}
\label{schmitdd}
\end{equation}
gives the diffusivity for salinity and other tracers.

