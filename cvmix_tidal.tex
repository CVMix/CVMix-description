\chapter{\scshape Energetically based mixing schemes from tidal dissipation}
\label{chapter:cvmix_tidal}

\minitoc
\vspace{.5cm}

\begin{mdframed}[backgroundcolor=lightgray!50]
This chapter summarizes the CVMix implementation of the parameterized
vertical mixing associated with tidal dissipation in both the ocean
interior and near the bottom.  The following CVMix Fortran module is
directly connected to the material in this chapter:
\begin{align*} 
 &{\tt cvmix\_tidal.F90}
\end{align*}
\end{mdframed}

\section{Introduction to tidal induced mixing}
\label{section:vert_tidal_intro}

Mixing arises when energy dissipates at the small scales in the
presence of a nonzero gradient of tracer and/or momentum.  There are
two sources of energy dissipation considered in this chapter.
\begin{itemize}

\item {\sc Internal waves in ocean interior}: Breaking internal
  gravity waves arise when barotropic tidal flow is scattered into
  internal tidal energy. This process occurs when astronomical tides
  interact with rough bottom topography,

\item {\sc Tidal waves interacting with continental shelves}:
  Frictional bottom drag is enhanced as tides encounter continental
  shelves (whose depths are generally 500~m or less).  There is an
  associated mixing of water masses due to this dissipation.

\end{itemize}
To resolve both of these dissipation processes explicitly in a
numerical model requires grid resolution no coarser than meters in the
vertical (throughout the water column), and 1-10 kilometers in the
horizontal.  This very fine resolution is not generally accessible to
global climate models, so it is necessary to consider a
parameterization.

CVMix has implementations for the following tide mixing
parameterizations.
\begin{itemize}
 
\item {\sc baroclinic or internal wave mixing}: \cite{Simmonsetal2004}
  presented the first implementation in an ocean climate model of an
  internal tide mixing parameterization.  \cite{Jayne2009} followed
  with an updated implementation.  A more recent study by
  \cite{Melet_etal_2013} implemented the ideas from \cite{Polzin2009}
  to remove the arbitrariness of the vertical deposition function used
  by \cite{Simmonsetal2004} and \cite{Jayne2009}.  Any of these
  schemes aim to provide a physically based replacement for the static
  vertical diffusivity of \cite{BryanLewis1979} (Chapter
  \ref{chapter:cvmix_background}).

\end{itemize}
Although CVMix provides an optional Prandtl number, it is general
practice to assume a unit Prandtl number for each of the tide
parameterization schemes.\footnote{The Prandlt number is the ratio of
  viscosity to diffusivity.}



\section{Energetic elements of tide mixing parameterizations}
\label{section:vert_tidal_formulation}

We now consider elements of how energetic based tide mixing
parameterizations are formulated.


\subsection{Bottom drag}
\label{subsection:bottom-drag-barotropic-tides}

Frictional bottom drag is typically parameterized as
\begin{equation}
  {\bf D}_{\mbox{\tiny bottom drag}} = C_{D} \, {\bf u} \, | {\bf u}|
  \qquad \mbox{(units of $\mbox{m}^{2}~\mbox{s}^{-2}$),}
\end{equation}
where $C_{D}$ is a dimensionless drag coefficient with a value on the
order of
\begin{equation}
C_{D} \approx  2 \times 10^{-3}. 
\end{equation}
Energy dissipation associated with this bottom drag is given by
\begin{equation}
 E_{\mbox{\tiny bottom drag}} =
  \rho_{o} \, {\bf u} \cdot {\bf D}_{\mbox{\tiny bottom drag}} 
 = \rho_{o} \, |{\bf u}|^{3}
\qquad \mbox{(units of $\mbox{W} \; \mbox{m}^{-2}$),}
\label{eq:energy-dissipation-bottom-drag-general}
\end{equation}
where $\rho_{o}$ is a reference ocean density.  

A component to the energy dissipation
(\ref{eq:energy-dissipation-bottom-drag-general}) is associated with
barotropic tides as they encounter the ocean bottom, particularly
continental shelves and other shallow ocean regions.  In an ocean
model that does not represent the astronomical tides, we may choose to
enhance the model's bottom velocity through a root-mean-square tidal
speed,$U_{\mbox{\tiny tide}}$, so that the bottom drag takes the form
\begin{equation}
  {\bf D}_{\mbox{\tiny bottom drag}} = 
  C_{D} \, {\bf u} \, \left( {\bf u}^{2} + U_{\mbox{\tiny tide}}^{2} \right)^{1/2},
\end{equation}
where now the velocity ${\bf u}$ refers to the model's resolved bottom
velocity field.  The modified energy dissipation from bottom drag
thus takes the form
\begin{equation}
 E_{\mbox{\tiny bottom drag}} =  \rho_{o} \, C_{D} \,  {\bf u}^{2} \, \left( {\bf u}^{2} + U_{\mbox{\tiny tide}}^{2} \right)^{1/2}.
\label{eq:energy-dissipation-bottom-drag-with-tides}
\end{equation}

Note that the tidal speed $U_{\mbox{\tiny tide}}$ may be a prescribed
global constant, as in the OCCAM simulations from \cite{Webbetal1998}.
It may instead be a map determined by an offline tide model, as in
\cite{Jayne_etal2001}.  Or, it may be computed online by periodically
running a tide model during the update of the primitive equation ocean
model.


\subsection{Wave drag from breaking internal gravity waves}

A drag associated with breaking internal gravity waves was written by
\cite{Jayne_etal2001} as
\begin{equation}
  {\bf D}_{\mbox{\tiny wave drag}} = (1/2) \, N_{\mbox{\tiny bott}}\,  
  \kappa_{\mbox{\tiny topo}} \, h_{\mbox{\tiny topo}}^{2} \, {\bf u}
\qquad \mbox{(units of $\mbox{m}^{2} \; \mbox{s}^{-2}$),}
\end{equation}
where $N_{\mbox{\tiny bott}}$ is the buoyancy frequency at the ocean
bottom, and $(\kappa_{\mbox{\tiny topo}},h_{\mbox{\tiny topo}})$ are
wavenumber (dimensions of inverse length) and amplitude (dimensions of
length) scales for the topography.  The product $\kappa_{\mbox{\tiny
    topo}} \, h_{\mbox{\tiny topo}}^{2}$ has dimensions of length and
defines a {\em roughness length}
\begin{equation}
  L_{\mbox{\tiny rough}} = \kappa_{\mbox{\tiny topo}} \, h_{\mbox{\tiny topo}}^{2}  
\label{eq:roughness-length}
\end{equation}
to be specified according to statistics of the observed ocean bottom
topography.  The internal wave drag can thus be written as
\begin{equation}
  {\bf D}_{\mbox{\tiny wave drag}} = (1/2) \, N_{\mbox{\tiny bott}}  \,  L_{\mbox{\tiny rough}} \, {\bf u}
\qquad \mbox{(units of $\mbox{m}^{2} \; \mbox{s}^{-2}$).}
\end{equation}

The energy dissipation associated with breaking internal gravity waves
is given by
\begin{equation}
\begin{split}
 E_{\mbox{\tiny wave drag}} &=
  \rho_{o} \, \langle {\bf u} \cdot {\bf D}_{\mbox{\tiny wave drag}} \rangle 
 \\
 &=
   (\rho_{o}/2) \, N_{\mbox{\tiny bott}}  \,  L_{\mbox{\tiny rough}}
   \,  \langle {\bf u}^{2} \rangle  
 \qquad \mbox{(units of $\mbox{W} \; \mbox{m}^{-2}$).}
\end{split}
\label{eq:wave-energy-flux}
\end{equation}
In the \cite{Jayne_etal2001} paper, they emphasize that
$\kappa_{\mbox{\tiny topo}}$, which sets the roughness length through
$L_{\mbox{\tiny rough}} = \kappa_{\mbox{\tiny topo}} \, h_{\mbox{\tiny
    topo}}^{2}$, is used as a tuning parameter, with the tide model
tuned to give sea level values agreeing with observations.  Then, the
energy dissipation can be diagnosed from the tide model.

As with the bottom drag (Section
\ref{subsection:bottom-drag-barotropic-tides}), the wave energy
dissipation arises from energy removed from the barotropic tides, yet
here the is transferred into baroclinic tides.  Some of the energy
transferred into the baroclinic tides dissipates locally due to local
wave breaking, and this then leads to enhanced mixing locally.  The
remaining baroclinic energy propogates away (i.e., it is non-local).
The ratio of local to non-local energy is not well known, and is the
focus of research.   


\subsection{Relating dissipation to mixing via \cite{Osborn1980}}

Mixing occurs when mechanical energy is dissipated in the presence of
stratification.  The relation between energy dissipation and mixing is
not known from first principles, so we consider dimensional arguments
to establish a useful form.  Since we are concerned with vertical
mixing, we assume that diffusivity is inversely proportional to the
vertical stratification, with stratification strength measured by the
buoyancy frequency, with a linearized form given by equation
(\ref{eq:buoyancy-frequency-elements})
\begin{equation}
N^{2} =  -\frac{g}{\rho} \, \left( 
   \frac{\partial \rho}{\partial \theta} \frac{\partial \theta}{\partial z}  
  +
   \frac{\partial \rho}{\partial S} \frac{\partial S}{\partial z}
\right), 
\end{equation}
and a fully nonlinear form detailed in Section
\ref{subsection:buoyancy-frequency-and-stability}.  Mechanical energy
per mass has units of $\mbox{m}^{2}~\mbox{s}^{-2} =
\mbox{J}~\mbox{kg}^{-1}$, and the dissipation of this energy, written
as $\epsilon$, has units of $\mbox{m}^{2}~\mbox{s}^{-3} =
\mbox{W}~\mbox{kg}^{-1}$
\begin{equation}
\epsilon = \mbox{mechanical energy dissipation in units of
 $\mbox{m}^{2}~\mbox{s}^{-3} = \mbox{W}~\mbox{kg}^{-1}$.}
\label{eq:energy-dissipation-epsilon}
\end{equation}
Together, the energy dissipation and buoyancy frequency define a
diffusivity given through the relation \citep{Osborn1980}
\begin{equation}
  \kappa_{\mbox{\tiny dissipate}} = \frac{\Gamma \, \epsilon}{N^{2}},
\label{eq:osborn-relation}
\end{equation} 
where the dimensionless parameter $\Gamma$ measures the efficiency
that mechanical energy dissipation translates into mixing that can be
parameterized by a diffusivity acting on vertical stratification.
This relation is used throughout the mixing community for converting
measurements of mechanical energy dissipation into diffusivity.

The efficiency parameter in equation (\ref{eq:osborn-relation}) is
often chosen as
\begin{equation}
  \Gamma = 0.2
\end{equation}
based measurements \citep{Osborn1980,Ivey_Imberger1991}.  However, in
regions of very weak vertical stratification, where $N^{2} \rightarrow
0$, we suggest following \cite{Melet_etal_2013}, in which the mixing
efficiency tends to zero according to
\begin{equation}
  \Gamma = 0.2 \left( \frac{N^{2}}{N^{2} + \Omega^{2}} \right)
\label{eq:gamma-stratification-dependent}
\end{equation} 
 where 
\begin{equation}
\begin{split}
 \Omega &=  \left(\frac{2\pi + 2\pi/365.24}{86400 \mbox{s}} \right)
 \\
 &=
 \left( \frac{\pi}{43082} \right) \mbox{s}^{-1} 
 \\
 &=
 7.2921 \times 10^{-5} \mbox{s}^{-1}.
\end{split}
\label{eq:Omega-defined}
\end{equation}
is the angular rotation rate of the earth about its axis and about the
sun.  This modified mixing efficiency reduces the regions where
spuriously large values of diffusivity may occur, especially next to
the bottom, where low values of $N^{2}$ may appear.  There is little
physical reason to believe the huge diffusivities diagnosed from
regions with $N^{2} < \Omega^{2}$.


\subsection{Vertical deposition function}
\label{subsection:vertical-deposition}

We are generally concerned in this chapter with mixing induced by
energy dissipation that is largest near the bottom.  This bottom
intensified dissipation leads to the largest levels of mixing also
near the bottom.  Yet there are means for dissipation to move upwards
into the water column, and it is this mixing that generally has far
more impact on the ocean stratification.  Details of how dissipation
moves upwards into the column remains a topic of research.  We present
here a formulation followed by the CVMix implementations of the
\cite{Simmonsetal2004} and \cite{Melet_etal_2013} schemes.  In this
case, we write the energy dissipation in the form
\begin{equation}
 \epsilon = {\cal E} \, F(z),
\label{eq:wave-drag-dissipation-form}
\end{equation}
where ${\cal E}$ is an energy dissipation times a length scale, and
$F(z)$ is a vertical deposition function with units of inverse length.
Both \cite{Simmonsetal2004} and \cite{Melet_etal_2013} chose
\begin{equation}
  {\cal E} =  \frac{ q \, E_{\mbox{\tiny wave drag}}(x,y)  }{\rho},
\label{eq:wave-dissipation-time-length}
\end{equation}
where $E_{\mbox{\tiny wave drag}}(x,y)$ is the energy input to wave
drag originating from the bottom (equation
(\ref{eq:wave-energy-flux})), $\rho$ is the {\it in situ} density, and
$q$ is the dimensionless fraction of energy that dissipates locally
rather than propagating away to dissipate non-locally. We have more to
say on $q$ in Section
\ref{subsection:local-or-non-local-energy-waves}.  The vertical
deposition function is assumed to integrate to unity over an ocean
column
\begin{equation}
  \int\limits_{-H}^{\eta} F(z) \, \mathrm{d}z = 1.  
\label{eq:normalization-of-Fz}
\end{equation}
\cite{Simmonsetal2004} chose an empirical exponential function
(equation (\ref{eq:simmons-vertical-function})) for $F(z)$, whereas
\cite{Melet_etal_2013} based their choice on theoretical results from
\cite{Polzin2009}.  


\subsection{Local versus non-local wave energy dissipation}
\label{subsection:local-or-non-local-energy-waves}

The dimensionless parameter, $q$, introduced in equation
(\ref{eq:wave-dissipation-time-length}) measures the fraction of wave
energy dissipated locally, and thus contributes to local mixing.
\cite{Simmonsetal2004} and \cite{Melet_etal_2013} both chose
\begin{equation}
  q = 1/3
\end{equation}  
based on the work of \cite{StLaurent_etal2002}.  The remaining 2/3 of
the wave energy propagates away and is assumed to dissipate
non-locally.  The non-local dissipation of internal tidal energy, as
well as the dissipation of internal energy from other sources (e.g.,
wind energy), are accounted for in an {\em ad hoc} manner via the
background diffusivity $\kappa_{0}$ (and background viscosity).  A
value within the range
\begin{equation}
  \kappa_{0} = (0.1 - 0.2) \times 10^{-4} \, \mbox{m}^{2} \, \mbox{s}^{-1}
\end{equation}
is recommended based on the measurements of \cite{Ledwell1993}.  Other
choices are considered in Chapter \ref{chapter:cvmix_background}.

Setting $q = 1/3$ globally is strictly incorrect for internal gravity
wave dissipation. The actual value is related to the modal content of
the excited internal tide, which is related to the roughness spectrum
of topography.  The redder the mode/roughness spectrum, the lower $q$.
For example, Hawaii has been modelled as a knife-edge by
\citep{StLaurent_etal2003}.  This topography excites predominantly low
modes, and these modes are stable, propogate quickly, and have long
interaction times.  That is, they propagate to the far field.
\cite{Klymak_etal2005} argue that $q=0.1$ for Hawaii from the Hawaiian
Ocean Mixing Experiment (HOME) data. For the mid-Atlantic ridge, the
use of $q=1/3$, as in \cite{Simmonsetal2004} and
\cite{Melet_etal_2013}, may be more suitable.

The bottom mixing scheme from \cite{Legg_eta2006} in effect assume 
\begin{equation}
 q=1   \qquad \mbox{bottom mixing scheme},
\end{equation}
which is sensible given that the mixing considered in their scheme
occurs predominantly within a bottom boundary layer.


\subsection{Prandtl number}

The Prandtl number is the ratio of viscosity to diffusivity.
\cite{Simmonsetal2004} do not discuss vertical viscosity in their
study.  If one considers a non-zero Prandtl number, then vertical
viscosity is enhanced along with the diffusivity when considering
internal wave breaking.  The following are examples of the Prandlt
number chosen for the tide mixing parameterizations.
\begin{itemize}

\item The earth system models of \cite{Dunne_etal_part1_2012} assume a
  unit Prandtl number for mixing related to tide mixing.

\item What about \cite{Jayne2009}?  

\end{itemize}


\subsection{General form of the vertical diffusivity}
\label{subsection:general-form-tide-diffusivity}

The previous considerations lead to the following general form for a
diffusivity arising from mechanical energy dissipation that originates
from the ocean bottom
\begin{subequations}
\begin{align}
 \kappa_{\mbox{\tiny dissipate}} &= 
 \frac{\Gamma \, \epsilon_{\mbox{\tiny dissipate}}}{N^{2}}
\\
 &=\frac{\Gamma \, {\cal E}_{\mbox{\tiny dissipate}} \, F(z) }{N^{2}}
\\
 &=
 \frac{q \, \Gamma \, E_{\mbox{\tiny dissipate}}(x,y) \,  F_{\mbox{\tiny dissipate}}(z)}  {\rho \, N^2}.
\end{align}
\label{eq:wave-vert-kappa-general-bottom}
\end{subequations}
The energy dissipation at the ocean bottom, $ E_{\mbox{\tiny
    dissipate}}(x,y)$, and the vertical deposition function,
$F_{\mbox{\tiny dissipate}}(z)$, distinguish the schemes considered by
\cite{Simmonsetal2004} and \cite{Melet_etal_2013}.


\subsection{Energetic balances}

One of the main reasons to formulate diffusivites based on mechanical
energy input is that this energy is exchanged in a conservative manner
within the ocean.  This conservation then leads to self-consistency
tests for the model implementation of various energy-based mixing
parameterizations.  We consider here in particular the work done
against stratification by vertical diffusion with a diffusivity
$\kappa_{\mbox{\tiny dissipate}}$, with this work given by
\begin{equation}
 {\cal P} \equiv  \int \kappa_{\mbox{\tiny dissipate}} \, \rho \, N^{2} \, \mathrm{d}V. 
\end{equation}
Use of equation (\ref{eq:wave-vert-kappa-simmons}) for the vertical
diffusivity with a constant mixing efficiency $\Gamma = 0.2$ yields
\begin{subequations}
\begin{align}
 {\cal P} &=  \int \kappa_{\mbox{\tiny dissipate}} \, \rho \, N^{2} \, \mathrm{d}V
 \\
 &= q \, \Gamma  \int  E_{\mbox{\tiny dissipate}}(x,y) \, \mathrm{d}x \, \mathrm{d}y,
\end{align}
\label{eq:energy-conserve-dissipation}
\end{subequations}
assuming $q \, \Gamma$ constant.  Note that to reach this result, we
set $\int F_{\mbox{\tiny dissipate}}(z) \, \mathrm{d}z = 1$ (Section
\ref{subsection:vertical-deposition}), which is a constraint that is
maintained by the CVMix implementation of the energetic-based mixing
schemes.  Equation (\ref{eq:energy-conserve-dissipation}) says that
the energy from some form of dissipation mechanism is deposited in the
ocean interior and works against stratification.

For the more general case of $q \, \Gamma$ spatially dependent, we
have the balance
\begin{subequations}
\begin{align}
 {\cal P} &=  \int \kappa_{\mbox{\tiny dissipate}}  \, \rho \, N^{2} \, \mathrm{d}V
 \\
 &= \int q \, \Gamma \,  E_{\mbox{\tiny dissipate}}(x,y) \, F_{\mbox{\tiny dissipate}}(z) \; \mathrm{d}V,
\end{align}
\label{eq:energy-conserve-wave-drag-again}
\end{subequations}
which again is a statement of energy conservation between wave
dissipation and mixing of density.  Although equation
(\ref{eq:energy-conserve-wave-drag-again}) is a trivial identity
following from the definition of the closure, it is not trivial to
maintain in the ocean model.  The main reason is that we work with
diffusivities when integrating the tracer equations in an ocean model,
and these diffusivities are often subjected to basic numerical
consistency criteria, such as the following.
\begin{itemize}

\item We may wish to have the diffusivities monotonically decay
  upwards in the column.  Given the $N^{-2}$ dependence of the
  diffusivity in equation (\ref{eq:wave-vert-kappa-general-bottom}),
  monotonicity is not guaranteed.  Without an added monotonicity
  constraint, the simulation can be subject to spurious instabilities
  in which intermediate depths become weakly stratified through some
  mixing, which in turn produces even larger diffusivities to further
  reduce the stratification.  \cite{Jayne2009} discovered this
  behaviour in his simulations with POP.

\item The diffusivities should be bounded by a reasonable number, such
  as $50-100 \, \mbox{cm}^{2} \, \mbox{sec}^{-1}$.
\end{itemize}
Imposing constraints such as these on the diffusivity corrupts the
identity (\ref{eq:energy-conserve-dissipation}).  In general, the
constraints remove energy from the interior, so that in practice $\int
\kappa_{\mbox{\tiny dissipate}} \, \rho \, N^{2} \, \mathrm{d}V < \int
q \, \Gamma \, E_{\mbox{\tiny dissipate}}(x,y) \, \mathrm{d}V$.


\section{The Simmons et al. (2004) scheme}
\label{section:simmons-etal}

To account for mixing associated with energy dissipation from breaking
internal gravity waves, \cite{Simmonsetal2004} propose a diffusivity
given by
\begin{subequations}
\begin{align}
 \kappa_{\mbox{\tiny simmons}} &= 
 \frac{\Gamma \, \epsilon_{\mbox{\tiny wave drag}}}{N^{2}}
\\
 &= \frac{\Gamma \, {\cal E}_{\mbox{\tiny wave drag}} \, F_{\mbox{\tiny simmons}}(z)}{N^{2}}
\\
 &=
 \frac{q \, \Gamma \, E_{\mbox{\tiny wave drag}}(x,y) \,  F_{\mbox{\tiny simmons}}(z) }  {\rho \, N^2},
\end{align}
\label{eq:wave-vert-kappa-simmons}
\end{subequations}
which again is the general form introduced in Section
\ref{subsection:general-form-tide-diffusivity}.  To reach this result,
we used equation (\ref{eq:wave-drag-dissipation-form}) to introduce
the vertical deposition function $F_{\mbox{\tiny simmons}}$, and
equation (\ref{eq:wave-dissipation-time-length}) to introduce the wave
drag energy dissipation, $E_{\mbox{\tiny wave drag}}$, given by
equation (\ref{eq:wave-energy-flux}).


\subsection{Calculation of the wave energy dissipation} 
\label{subsection:wave-drag-details-simmons}

The wave energy dissipation, $E_{\mbox{\tiny wave drag}}$, is
evaluated as follows.
\begin{itemize}

\item $N_{\mbox{\tiny bott}}$ is computed from the model's evolving
  buoyancy frequency at the top face of a bottom boundary layer (often
  just the bottom-most tracer cell).  Note that the buoyancy frequency
  at the bottom face of the bottom-most cell is zero, by definition.

\item The effective roughness length $L_{\mbox{\tiny rough}} =
  \kappa_{\mbox{\tiny topo}} \, h_{\mbox{\tiny topo}}^{2}$ requires an
  algorithm to compute $h_{\mbox{\tiny topo}}$ from observed bottom
  topography, and tide model to tune $\kappa_{\mbox{\tiny topo}}$.
  However, in practice what can be done is to take $h_{\mbox{\tiny
      topo}}$ given some variance of topography within a grid cell,
  and then tune $E_{\mbox{\tiny wave drag}}$ to be roughly 1TW in
  ocean deeper than 1000m, with $\kappa_{\mbox{\tiny topo}}$ as the
  tuning paramter.

\end{itemize}


\subsection{Deposition function}

The bottom intensified vertical profile, or deposition function, is
taken as
\begin{subequations}
\begin{align}
  F_{\mbox{\tiny simmons}}(z) &= \frac{e^{-(d-h)/\zeta}}{\zeta \, (1 - e^{-d/\zeta} )}
  \\
  &= \frac{e^{h/\zeta}}{\zeta \, (e^{d/\zeta}-1)}
 \\
  &= e^{-z/\zeta} \left( \frac{ e^{\eta/\zeta} }{ \zeta \, ( e^{d/\zeta} - 1 ) } \right).
\end{align}
\label{eq:simmons-vertical-function}
\end{subequations}
 In this expression,
\begin{equation}
  d = H + \eta
\end{equation}
is the time dependent thickness of water between the free surface at
$z=\eta$ and the ocean bottom at $z=-H$, and
\begin{equation}
  h=-z+\eta
\end{equation}
is the time dependent distance from the free surface to a point within
the water column.\footnote{We emphasize that with a free surface, $d$
and $h$ are generally time dependent.  Furthermore, with general
vertical coordinates, $h$ is time dependent for all grid cells.}  The
chosen form of the deposition function is motivated by the
microstructure measurements of \cite{StLaurent_etal2001} in the
abyssal Brazil Basin, and the continental slope measurements of
\cite{Moum_etal2002}.  This profile respects the observation that
mixing from breaking internal gravity waves, generated by scattered
barotropic tidal energy, is exponentially trapped within a distance
$\zeta$ from the bottom.  An {\em ad hoc} decay scale of
\begin{equation}
  \zeta = 500 \, \mbox{m}
\end{equation}
is suggested by \cite{Simmonsetal2004} for use with internal gravity
wave breaking in the abyssal ocean.


\subsection{Regularization of the diffusivity}

The diffusivities resulting from this parameterization can reach
levels upwards of the maximum around $20 \times 10^{-4} \,
\mbox{m}^{2} \, \mbox{s}^{-1}$ seen in the \cite{Polzinetal97}
results.  Due to numerical resolution issues, the scheme can in
practice produce even larger values.  We need to consider the physical
relevance of these large values.  The following lists some options
that the modeller may wish to exercise.

\begin{itemize}

\item We may choose to limit the diffusivity to be no larger than a
  maximum value, defaulted to $50 \times 10^{-4} \, \mbox{m}^{2} \,
  \mbox{s}^{-1}$ in CVMix.

\item Based on observations, the mechanical energy input from wave
  drag (equation (\ref{eq:wave-energy-flux})) should not exceed
  roughly $0.1 \, \mbox{W} \, \mbox{m}^{-2}$ at a grid point (Bob
  Hallberg, personal communication 2008).  Depending on details of the
  bottom roughness and tide velocity amplitude, a typical model
  implementation may easily exceed this bound.  Hence, it may be
  necessary to cap the mechanical energy input to be no larger than a
  set bound.

\item Use of the stratification dependent mixing efficiency
  (\ref{eq:gamma-stratification-dependent}) provides a physically
  based means to regularize the regions where $N^{2}$ can get
  extremely small. 

\end{itemize}


\subsection{Regarding a shallow depth cutoff}

\cite{Simmonsetal2004} do not apply their scheme in waters with ocean
bottom shallower than 1000m, whereas \cite{Jayne2009} applies the
scheme for all depths.  CVMix has a namelist that allows for setting
a cutoff depth.  In principle, there is nothing wrong with using the
\cite{Simmonsetal2004} scheme all the way to shallow waters, and
removing the somewhat arbitrary depth cutoff is more satisfying.  So
one may wish to naively use $q = 1/3$ without a 1000m depth cutoff.

Likewise, $\zeta = 500$m globally may be a reasonable choice. The
structure function will properly integrate to unity, whether or not
the ocean depth $H$ is greater or less than $\zeta$.

\subsection{Further comments} 

Here are some further points to consider when setting some of the
namelists for this scheme.
\begin{itemize}

\item One means to ensure that the diffusivities are within a
  reasonable bound, without capping them after their computation, is
  to artificially restrict the stratification used in the calculation
  to be no less than a certain number.  \cite{Simmonsetal2004} chose
  the floor value $N^{2} \ge 10^{-8} \, \mbox{s}^{-2}$.  There is a
  great deal of sensitivity to the floor value used.  GFDL practice is
  to keep the floor value quite low so that $N^{2}_{\mbox{\tiny min}}
  < \Omega^{2}$.

\item If the maximum diffusivity realized by the scheme is allowed to
  be very large, say much greater than as $1000 \, \mbox{cm}^{2} \,
  \mbox{sec}^{-1}$, then the near bottom stratification can become
  very small.  In this case, $E_{\mbox{\tiny wave drag}}$ can dip
  below the canonical 1TW value.  This process resembles a negative
  feedback in some manner, though it has not been explored
  extensively.

\end{itemize}


% \section{The Melet et al. (2013) scheme}
% \label{section:melet-etal}

% A limitation of the \cite{Simmonsetal2004} scheme is the arbitrary
% choice of their empirical vertical deposition function
% (\ref{eq:simmons-vertical-function}) and the corresponding exponential
% decay length, $\zeta$.  \cite{Melet_etal_2013} build on ideas proposed
% by \cite{Polzin2004,Polzin2009} to overcome these limitations.  In
% their parameterization, they propose a deposition function
% corresponding to finescale internal wave shear producing an energy
% dissipation given by
% \begin{equation}
%  \epsilon_{\mbox{\tiny melet}} = 
%  \left( \frac{\epsilon_{o}}{ (1 + z^{*}/z_{p}^{*})^{2}} \right)
%  \, \left( \frac{N^{2}(z)}{\overline{N^{2}}^{z}} \right) 
%  \, \left( \frac{1}{H} + \frac{1}{z_{p}^{*}} \right).
% \end{equation}
% The corresponding diffusivity is given by the general form
% (\ref{eq:wave-vert-kappa-general-bottom}), so that 
% \begin{equation}
% \begin{split}
%  \kappa_{\mbox{\tiny melet}} &= 
%  \frac{\epsilon_{\mbox{\tiny melet}} \, \Gamma}{N^{2}}
% \\
%  &= 
%  \left( \frac{ {\cal E}_{\mbox{\tiny wave drag}}}{ (1 + z^{*}/z_{p}^{*})^{2}} \right)
%  \, \left( \frac{\Gamma}{\overline{N^{2}}^{z}} \right) 
%  \, \left( \frac{1}{H} + \frac{1}{z_{p}^{*}} \right)
% \\
%  &=
%  \left( \frac{q \, \Gamma \, E_{\mbox{\tiny wave drag}}}{\rho \, \overline{N^{2}}^{z}} \right)
%  \left( \frac{1}{ (1 + z^{*}/z_{p}^{*})^{2}} \right)
% \, \left( \frac{1}{H} + \frac{1}{z_{p}^{*}} \right),
% \end{split}
% \end{equation}
%  where we set 
% \begin{equation}
% {\cal E}_{\mbox{\tiny wave drag}} = \frac{q \,  E_{\mbox{\tiny wave drag}}}{\rho}
% \end{equation}
% according to equation (\ref{eq:wave-dissipation-time-length}).  

% The vertical deposition function
% \begin{equation}
%  F_{\mbox{\tiny melet}} = \frac{N^{2}}{ \overline{N^{2}}^{z}} 
%  \, \left( \frac{1}{ (1 + z^{*}/z_{p}^{*})^{2}} \right)
%  \, \left( \frac{1}{H} + \frac{1}{z_{p}^{*}} \right)
% \end{equation}
% is algebraic, rather than the exponential suggested by
% \cite{Simmonsetal2004} (equation
% (\ref{eq:simmons-vertical-function})).  A fundamental element to the
% deposition function is the scaled vertical distance from the bottom
% \begin{equation}
%  z^{*}( h_{\mbox{\tiny bott}}) = 
%  \frac{1}{\overline{N^{2}}^{z}}
%  \int\limits_{0}^{ h_{\mbox{\tiny bott}}} N^{2}(z')  \, \mathrm{d}z',
% \end{equation}
% where the vertical integral extends from the bottom at $h_{\mbox{\tiny
%     bott}} = 0$ to an arbitrary distance above the bottom.  The depth
% averaged squared buoyancy frequency is given by
% \begin{equation}
%  \overline{N^{2}}^{z} = \frac{1}{H + \eta} \, \int\limits_{-H}^{\eta} N^{2}(z) \, \mathrm{d}z. 
% \end{equation}
% Note that by definition, the rescaled vertical height from the bottom
% satisfies
% \begin{equation}
%   0 \le z^{*} \le H.
% \end{equation}
% We also introduced the rescaled length scale according to 
% \begin{equation}
%  z_{p}^{*}  = z_{p} \, \left( \frac{N^{2}_{\mbox{\tiny bott}}}{ \overline{N^{2}}^{z}} \right),
% \end{equation}
% The bottom buoyancy frequency, $N^{2}_{\mbox{\tiny bott}}$, is
% computed according to the discussion in Section
% \ref{subsection:wave-drag-details-simmons}.

% The length scale $z_{p}$ is computed according to
% \cite{Polzin2004,Polzin2009}, and needs to be written here...




